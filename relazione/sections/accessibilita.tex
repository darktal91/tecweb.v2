Per garantire che il sito sia accessibile al maggior numero di utenti e dispositivi è stato realizzato tenendo in considerazione le linee guida del WCAG. \newline

\subsection{Immagini}
L'uso di immagini come contenuto è stato estremamente limitato. In ogni caso per ognuna delle immagini di contenuto utilizzate è stato fornito un attributo \textit{alt} non vuoto contenente una descrizione dell'immagine. Il corretto funzionamento degli attributi \textit{alt} è stato verificato usando l'estensione per chrome \textit{Image Alt Text Viewer}.

Per la mappa dei padiglioni, presente nella pagina \textit{Padiglioni}, non è stata fornita una descrizione dettagliata perché l'immagine non offre particolari informazioni aggiuntive rispetto all'elenco testuale presente di fianco, è solo un supporto visivo aggiuntivo per orientarsi. Inoltre, una descrizione dettagliata della mappa avrebbe richiesto l'utilizzo dell'attributo \textit{longdesc}, il quale però è scarsamente supportato dai browser.

\subsection{Colori e contrasto}
La scelta dei colori per il sito è stata fatta tenendo in considerazione la presenza di diffusi problemi visivi tra la popolazione, e quindi probabilmente anche tra gli utenti. I colori sono stati scelti in modo da garantire sempre un contrasto elevato tra le varie parti. Per verificare il contrasto tra i colori del testo e i quelli dello sfondo nella parte di contenuto delle pagine è stato usato il sito \url{https://snook.ca/technical/colour_contrast/colour.html}, che ha dato esito positivo.

Inoltre, utilizzando l'estensione \textit{Spectrum} per il browser chorome è stato testato il comportamento della grafica del sito in caso di vari tipi di daltonismo. I risultati sono stati buoni in alcuni casi e accettabili in altri. Un unico elemento del sito è risultato problematico per un tipo di daltonismo: il link alla home nella breadcrumb. Purtroppo, a causa di mancanza di tempo, non è stato possibile risolvere questo problema, perché il colore di quel link è conforme al colore di tutti i link di contenuto del sito, e cambiare la colorazione della barra che contiene la bradcrumb avrebbe richiesto di cambiare anche quasi tutti gli altri colori della grafica. Tuttavia questo problema non è particolarmente grave: l'informazione che si perde non è di vitale importanza considerando la struttura molto semplice del sito e la presenza di un link alla home page nel menù sottostante.

\subsection{Markup e fogli di stile}
Per garantire la massima compatibilità possibile con i vari dispositivi il sito è stato realizzato rispettando lo standard XHTML Strict e le linee guida W3C. Tutte le pagine sono state validate utilizzando il validatore del W3C (\url{https://validator.w3.org}). \newline
I fogli di stile sono stati realizzati utilizzando CSS puri e validati usando il sito \url{https://jigsaw.w3.org/css-validator/}.

\subsection{Lingua}
La lingua principale del sito è l'italiano. Nei casi in cui sono stati inseriti termini non appartenenti alla lingua italiana essi sono stati rinchiusi in un marcatore che ne dichiarasse la lingua di appartenenza, ad esempio <span xml:lang="en">...</span>. Questo accorgimento serve per permettere agli screen reader di leggere tali termini utilizzando la corretta pronuncia.

\subsection{Tabelle}
Nel sito è stata utilizzata una sola tabella nella pagina \textit{Controllo Ricavi}, destinata al solo utente amministratore. Per renderla accessibile è stato specificato attentamente l’attributo \textit{"summary"} ed inserito l’attributo \textit{"scope=col"} negli header. \newline
Per quanto riguarda l'attributo "summary", è stato preso come modello l'esempio realizzato da Michele Diodati nel suo libro \textit{Accessibilità - Guida Completa}, al capitolo 8 (Punto di controllo 5.5, priorità 3). \newline
È stata inoltre verificata la corretta visualizzazione della tabella provando su più browser e dispositivi.

\subsection{Compatibilità e portabilità}
Il sito è stato progettato e realizzato con l'obiettivo di essere usabile dal maggior numero di utenti possibile, anche se non provvisti di browser di ultima generazione. 
Sono stati effettuati test con i seguenti browser desktop:
\begin{itemize}
	\item Internet Explorer 8;
	\item Internet Explorer 11;
	\item Edge versione 25;
	\item Chrome versione 52;
	\item Mozilla Firefox versione 41.
\end{itemize}
E sui seguenti browser mobile:
\begin{itemize}
	\item Silk versione 52 (browser Kindle Fire);
	\item Chrome per Android versione 52.
\end{itemize}
Inoltre, è stata testata la navigazione utilizzando il browser testuale Lynx.

\subsection{Dinamicità}
Per evitare di creare problemi o disturbo agli utenti con testi o immagini in movimento tutti i contenuti del sito sono statici.

\subsection{Indipendenza dal dispositivo}
Il sito è stato progettato e realizzato con l'obiettivo di essere agevolmente navigabile sia tramite computer che tramite dispositivi mobile (smartphone e tablet).
I membri del gruppo hanno testato il sito utilizzando tutti i dispositivi in nostro possesso ed è risultato visitabile in tutti i casi. \newline
In ogni caso la struttura di HTML e CSS è stata realizzata seguendo le linee guida di WCAG per l'accessibilità, quindi è ragionevole supporre che il sito sia visitabile con qualsiasi dispositivo.

\subsection{Contesto e aiuto per l'orientamento}
Anche se la struttura del sito è molto semplice, in ogni pagina è presente una \textit{breadcrumb} che permette all'utente di sapere sempre in che posizione si trova all'interno del sito. Inoltre, il titolo di ogni pagina identifica chiaramente la stessa. \newline
All'interno del contenuto sono stati inseriti dei link, dove necessario, per permettere all'utente di spostarsi agevolmente all'interno del sito. Ad esempio nelle pagine in cui è richiesta l'autenticazione per svolgere alcune operazioni è presente, accanto alla spiegazione della necessità dell'autenticazione, un collegamento alla pagina di login.

\subsection{Link}
Per quanto riguarda il testo dei link, si è cercato di renderlo il più chiaro possibile anche al di fuori del suo contesto

Per la colorazione dei link all'interno dei contenuti del sito è stato scelto di utilizzare dei colori diversi da quelli standard per ottenere una maggior integrazione con il resto della grafica. I link visitati cambiano colore.

I link del menù, invece, non cambiano colore quando visitati. Questa scelta, benché non ottimale e non in linea con le raccomandazioni, non dovrebbe comportare disorientamento nell'utente durante la navigazione perché il numero di pagine è estremamente ridotto, la struttura del sito è molto semplice e inoltre è presente una breadcrumb.

\subsection{Screen reader}
Nell'analisi delle caratteristiche degli utenti a cui il sito è indirizzato non si è ritenuta significativa la fascia di utenti non vedenti, in quanto la saga di Star Wars deve buona parte del suo successo all'impatto visivo ed agli effetti speciali. \newline
Tuttavia, il sito è risultato perfettamente usabile tramite browser testuale, ed è stato realizzato seguendo le linee guida di accessibilità, quindi ci si aspetta che risulti navigabile anche utilizzando uno screen reader.
