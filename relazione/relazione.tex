\documentclass[12pt,a4paper]{article}

\usepackage{lmodern}
\usepackage[T1]{fontenc}
\usepackage[utf8]{inputenc}
\usepackage[italian]{babel}

\usepackage{hyperref}
\usepackage{graphicx}
\graphicspath{{img/}}
\usepackage{float}

\usepackage{titlesec}

\setcounter{secnumdepth}{4}

\titleformat{\paragraph}
{\normalfont\normalsize\bfseries}{\theparagraph}{1em}{}
\titlespacing*{\paragraph}
{0pt}{3.25ex plus 1ex minus .2ex}{1.5ex plus .2ex}

\begin{document}
\begin{titlepage}
	\centering
	\vspace*{1cm}
	{\scshape\LARGE \textbf{EmpireCon} \par}
	\vspace{0.5cm}
	{\Large \textbf{\textit{Relazione sul progetto per il corso di Tecnologie Web}}\par}
	\vspace{3cm}
	{\Large Andrea Cardin, 1030310\par}
	{\Large Andrea Nalesso, 1026100\par}
	{\Large Ismaele Gobbo, 1028902\par}
	\vspace{6cm}
	{\normalsize \textbf{Indirizzo del sito:} \url{boh}\par}
	{\normalsize \textbf{Utente amministratore:} admin\par}
	{\normalsize \textbf{Password amministratore:} password\par}
	{\normalsize \textbf{Utente comune:} \textbf{TODO}\par}
	{\normalsize \textbf{Password utente comune:} \textbf{TODO}\par}
	{\normalsize \textbf{Referente del gruppo:} darktal91@gmail.com (Andrea Cardin)\par}
	\vspace*{\fill}
\end{titlepage}

\tableofcontents
\newpage

% Come inserire una sezione
% \section{Nome sezione}
% \input{sections/nome file latex della sezione senza .tex}
% \newpage
% Esempio:
% \section{Introduzione}
% \subsection{Abstract}
Lo scopo del progetto è quello di realizzare un sito per l'Empire Con, una ipotetica convention di Star Wars che si svolge presso la Morte Nera II, la quale si trova in orbita attorno a Endor.

L'obiettivo del sito è quello di permettere agli aspiranti visitatori di poter acquistare i biglietti per l'accesso alla convention, lasciare dei commenti in un guestbook, ricevere delle news, conoscere quali eventi che si svolgeranno e in quale padiglione.

Abbiamo inoltre previsto per l'amministratore del sito la possibilità aggiungere nuovi eventi, modificare o eliminare gli eventi esistenti, di controllare i ricavi generati dalla vendita dei biglietti e di moderare la sezione commenti.

\subsection{Caratteristiche dell’utenza}
Le persone che potrebbero essere interessate a partecipare ad una convention su Star Wars sono i fan della sega ed è a loro che ci rivolgiamo: abbiamo quindi preso come assunto che i nostri visitatori siano in possesso di una conoscenza almeno minima del mondo di Star Wars e della sua terminologia; eventuali conoscenze mancanti non pregiudicano però la fruizione del sito.

Star Wars è una saga in circolazione da ormai 40 anni e quindi ci aspettiamo un'utenza omogenea negli interessi, ma non per l'età: si va infatti dal ragazzo adolescente all'adulto ormai over 60.
Abbiamo quindi deciso di non adottare una differenziazione in base ai contenuti, ma di puntare su una interfaccia che sia semplice.

\subsection{Software utilizzato, alcune scelte progettuali ed installazione}
Per poter lavorare assieme avendo una pila software comune abbiamo utilizzato docker: questo ci ha permesso di ridurre al minimo l'overhead, in quanto una soluzione basata su macchine virtuali sarebbe rivelata troppo esosa in termini di risorse per la macchina di uno dei componenti del gruppo; mentre lavorare esclusivamente utilizzando i server di tecnologie-web sarebbe stato meno pratico.
Docker è progetto open-source che automatizza il deployment di applicazioni all'interno di container software.
La pila software del server di tecnologie è stata quindi replicata fedelmente e la relativa immagine di docker condivisa tra i membri del gruppo è stata condivisa grazie ad un repository di immagini: è stato possibile quindi apportare modifiche all'immagine e condividerle in maniera semplice, in quanto era richiesto agli altri membri semplicemente di lanciare un comando da terminale.
Non possiamo che dirci soddisfatti della scelta di adottare docker e ci sentiamo di poter suggerire la stessa scelta anche agli altri gruppi, anche nel caso in cui fosse possibile utilizzare una macchina virtuale in comune o effettuare una "installazione nativa" dell'intera pila software.
Per lavorare in maniera concorrente abbiamo utilizzato come sistema di versionamento git ed adottato un workflow per quanto riguarda il branching: sebbene all'inizio non risultasse molto naturale e richiedesse qualche sforzo e conoscenza in più, alla lunga è stato apprezzato.
Abbiamo inoltre utilizzato una serie di script per automatizzare alcune operazioni ripetitive: principalmente per gestire i permessi, per testare la validità rispetto agli schema dei nostri file XML e per gestire docker.
Come prova del 9 abbiamo testato utilizzando il server di tecnologie web a progetto ormai concluso, prima di stilare la relazione.
Per installare su uno qualsiasi degli userspace del server di laboratorio, è necessario:
% \newpage

\section{Introduzione}
\subsection{Abstract}
Lo scopo del progetto è quello di realizzare un sito per l'Empire Con, una ipotetica convention di Star Wars che si svolge presso la Morte Nera II, la quale si trova in orbita attorno a Endor.

L'obiettivo del sito è quello di permettere agli aspiranti visitatori di poter acquistare i biglietti per l'accesso alla convention, lasciare dei commenti in un guestbook, ricevere delle news, conoscere quali eventi che si svolgeranno e in quale padiglione.

Abbiamo inoltre previsto per l'amministratore del sito la possibilità aggiungere nuovi eventi, modificare o eliminare gli eventi esistenti, di controllare i ricavi generati dalla vendita dei biglietti e di moderare la sezione commenti.

\subsection{Caratteristiche dell’utenza}
Le persone che potrebbero essere interessate a partecipare ad una convention su Star Wars sono i fan della sega ed è a loro che ci rivolgiamo: abbiamo quindi preso come assunto che i nostri visitatori siano in possesso di una conoscenza almeno minima del mondo di Star Wars e della sua terminologia; eventuali conoscenze mancanti non pregiudicano però la fruizione del sito.

Star Wars è una saga in circolazione da ormai 40 anni e quindi ci aspettiamo un'utenza omogenea negli interessi, ma non per l'età: si va infatti dal ragazzo adolescente all'adulto ormai over 60.
Abbiamo quindi deciso di non adottare una differenziazione in base ai contenuti, ma di puntare su una interfaccia che sia semplice.

\subsection{Software utilizzato, alcune scelte progettuali ed installazione}
Per poter lavorare assieme avendo una pila software comune abbiamo utilizzato docker: questo ci ha permesso di ridurre al minimo l'overhead, in quanto una soluzione basata su macchine virtuali sarebbe rivelata troppo esosa in termini di risorse per la macchina di uno dei componenti del gruppo; mentre lavorare esclusivamente utilizzando i server di tecnologie-web sarebbe stato meno pratico.
Docker è progetto open-source che automatizza il deployment di applicazioni all'interno di container software.
La pila software del server di tecnologie è stata quindi replicata fedelmente e la relativa immagine di docker condivisa tra i membri del gruppo è stata condivisa grazie ad un repository di immagini: è stato possibile quindi apportare modifiche all'immagine e condividerle in maniera semplice, in quanto era richiesto agli altri membri semplicemente di lanciare un comando da terminale.
Non possiamo che dirci soddisfatti della scelta di adottare docker e ci sentiamo di poter suggerire la stessa scelta anche agli altri gruppi, anche nel caso in cui fosse possibile utilizzare una macchina virtuale in comune o effettuare una "installazione nativa" dell'intera pila software.
Per lavorare in maniera concorrente abbiamo utilizzato come sistema di versionamento git ed adottato un workflow per quanto riguarda il branching: sebbene all'inizio non risultasse molto naturale e richiedesse qualche sforzo e conoscenza in più, alla lunga è stato apprezzato.
Abbiamo inoltre utilizzato una serie di script per automatizzare alcune operazioni ripetitive: principalmente per gestire i permessi, per testare la validità rispetto agli schema dei nostri file XML e per gestire docker.
Come prova del 9 abbiamo testato utilizzando il server di tecnologie web a progetto ormai concluso, prima di stilare la relazione.
Per installare su uno qualsiasi degli userspace del server di laboratorio, è necessario:

\section{Progettazione}
\subsection{Struttura del sito}
Ogni pagina del sito è formata dai seguenti elementi:
\begin{itemize}
	\item Header;
	\item breadcrumb;
	\item box che permette il login o mostra l'account con cui si è autenticati;
	\item menù;
	\item contenuto della pagina;
	\item footer.
\end{itemize}

\subsubsection{Mappa del sito}
\begin{figure}[H]
	\centering
	\includegraphics[width=16cm]{mappa-sito-small}
	\caption{Mappa del sito.}
\end{figure}
Oltre alla home page, che è concettualmente la radice del sito ed è ovviamente accessibile a tutti gli utenti, le varie pagine sono concettualmente suddivise nelle seguenti aree basate sull'attore che le andrà ad utilizzare:
\begin{itemize}
	\item Per soli utenti non autenticati;
	\item per soli utenti autenticati;
	\item per tutti gli utenti;
	\item per il solo utente amministratore.
\end{itemize}

\paragraph{Home page}
È, concettualmente, la radice del sito, nonché la pagina in cui l'utente si dovrebbe ritrovare quando si collega al sito. Contiene un paragrafo introduttivo di benvenuto e presentazione, seguito dalle news riguardante l'evento.

\paragraph{Per soli utenti non autenticati}
Quest'area contiene le pagine destinate agli utenti non autenticati, permettendo loro di interagire con le funzioni di autenticazione e registrazione al sito.
Pagine che ne fanno parte:
\begin{itemize}
	\item Login;
	\item Registrazione.
\end{itemize}

\paragraph{Per soli utenti autenticati}
Quest'area contiene le pagine che possono essere visitate o accedute solo dagli utenti che hanno eseguito l'autenticazione al sito. Le pagine di questa area permettono di interagire con gli aspetti relativi alla gestione del proprio account, come ad esempio la modifica dei dati personali o della password, la possibilità di effettuare il logout e, infine, un elenco dei biglietti acquistati da quell'account. L'accesso ad una di queste pagine da parte di un utente non autenticato produce un errore oppure, nel caso di area utente e logout, un reindirizzamento. Inoltre, ad un utente non autenticato non sono mai mostrati collegamenti a queste pagine.
Pagine che ne fanno parte:
\begin{itemize}
	\item Area Utente;
	\item Logout;
	\item Biglietti Acquistati.
\end{itemize}

\paragraph{Per tutti gli utenti}
Queste pagine sono accessibili a tutti gli utenti, a prescindere dal loro stato di autenticazione, tramite il menù.
Contengono i contenuti veri e propri del sito, con informazioni sulla fiera, l'acquisto dei biglietti e la possibilità di lasciare dei commenti.
Pagine che ne fanno parte:
\begin{itemize}
	\item Eventi:
		Presenta la lista di tutti gli eventi che si terranno nel corso della fiera. Per ogni evento vengono offerte informazioni utili quali la data, l'ora di inizio, l'ora di fine, il padiglione in cui si svolgeranno ed una descrizione.
		L'utente amministratore ha la possibilità di svolgere alcune funzioni aggiuntive: l'inserimento di un nuovo evento e la modifica o cancellazione di un evento già inserito.
	\item Padiglioni:
		Mostra l'elenco dei padiglioni presenti nella fiera suddivisi per settori. Inoltre mostra anche una mappa con la posizione di ogni padiglione. Il contenuto è uguale per tutte le categorie di utenti.
	\item Acquista biglietti:
		Elenca a tutti i tipi di utenti le varie tipologie di biglietti disponibili ed il loro prezzo. Ad un utente autenticato offre anche la possibilità di procedere all'acquisto di uno o più biglietti.
	\item Commenti:
		Rappresenta il "libro ospiti" del sito. Gli utenti di tutte le categorie possono leggere i commenti. Un utente autenticato ha a disposizione la funzione di eliminazione per i commenti da lui inviati, mentre l'utente amministratore può eliminare qualsiasi commento.
	\item Dove siamo:
		Offre indicazioni per raggiungere il luogo della fiera. Il contenuto è uguale per tutte le categorie di utenti.
	\item Contatti:
		Contiene informazioni su come contattare gli organizzatori della fiera. Il contenuto è uguale per tutte le categorie di utenti.
\end{itemize}

\paragraph{Per il solo utente amministratore}
A questa categoria appartiene una sola pagina: Controllo ricavi.
Questa pagina è destinata all'utente amministratore e presenta in forma tabellare il resoconto sui biglietti venduti e sui guadagni.

\subsection{Layout}
Per quanto riguarda il layout del sito, è stato scelto di adottare schema a tre pannelli (con breadcrumb), costituito da un header per l’intestazione del sito e la barra di navigazione, un footer, un menu di navigazione a sinistra: questo si riflette bene nell'organizzazione del nostro codice a template.
Sono state inoltre previste due tipologie principali di visualizzazione: una per i normali schermi di desktop/notebook ed una per i dispositivi mobile (o comunque con lo schermo di risoluzione ridotta). \newline
La versione mobile presenta un layout verticale, dove viene visualizzato prima l'header, poi il menu, quindi il body, ed infine il footer.

\subsection{Scelte progettuali}
Dato che la fascia d'utenza individuata risulta essere disomogenea in fatto di età e prevedendo utenti con più di 50 anni, è stato pensato di rendere il sito il più semplice ed intuitivo possibile. Di conseguenza, si è cercato di mantenere il design quanto più semplice e minimale possibile, e suddividere il sito in un numero relativamente limitato di pagine. Inoltre, ogni pagina ha un nome che la identifica in modo chiaro e ha dei contenuti specifici e strettamente correlati al titolo. Per agevolare la navigazione sono stati inseriti dei link dove necessario per completare operazioni evitando disorientamento, come ad esempio un link alla pagina di login di fianco al messaggio che informa l'utente della necessità di autenticarsi per usufruire di una funzione del sito. \newline
Infine, data la sempre crescente diffusione dei dispositivi mobile, è stata posta particolare attenzione nel rendere il sito fruibile anche utilizzando questi dispositivi. 
	

\section{Realizzazione}
\subsection{Perl}
Per la componente dinamica e di back-end del sito è stato utilizzato il linguaggio Perl come richiesto dalle specifiche del progetto. La versione utilizzata è quella installata nel server tecweb dell'Università. \newline
Le librerie principali utilizzate, dopo una iniziale valutazione tra quelle disponibili, sono state:
\begin{itemize}
	\item \textbf{CGI:} Per la gestione delle connessioni, il passaggio dei parametri HTTP tra le pagine e la gestione delle sessioni;
	\item \textbf{LibXML:} Per la gestione e manipolazione dei file XML;
	\item \textbf{HTML::Template:} Per la costruzione e presentazione dei contenuti. 
\end{itemize}
Tutte le pagine del sito sono dinamiche e realizzate in Perl per poter gestire le sessioni e le operazioni degli utenti.
Ogni pagina è realizzata in maniera modulare tramite template innestati. L'elemento principale è il template \textit{Page} che contiene il menù e il breadcrumb del sito, nonché richiama l'inclusione di altri template: \textit{Header}, che contiene il blocco \textit{head} della pagina, ed il \textit{Footer}. Oltre ai template sopraccitati, che sono presenti in ogni pagina e adattati tramite variabili dinamiche, il template Page richiede l'inclusione del contenuto vero e proprio della pagina, che è stato realizzato sotto forma di un template specifico per ogni pagina del sito. La dinamicità dei contenuti è ottenuta tramite l'interazione (sotto forma di variabili) tra il template del contenuto e la corrispondente pagina cgi.
La libreria LibXML è stata scelta perché è risultata essere la più completa e di più semplice utilizzo.

\subsection{XML}
Il progetto include quattro file XML utilizzati per salvare in modo persistente i dati. Per ogni file XML è stato realizzato un XMLSchema che ne definisce la struttura. Gli schemi sono stati realizzati utilizzando il modello \textit{Tende alla Veneziana}. Utilizzando il tool \textit{xmllint} gli schemi sono stati validati contro lo schema che definisce gli XMLSchema fornito dal W3C, mentre i file XML sono stati validati contro i rispettivi schemi da noi scritti. \newline
I file XML realizzati sono:
\begin{itemize}
	\item \textbf{utenti:} contiene tutti i dati degli utenti che si registrano al sito, compresi username e password necessari per l'autenticazione. La password viene cifrata con una funzione di hash prima di essere salvata;
	\item \textbf{commenti:} contiene tutti i commenti inviati dagli utenti;
	\item \textbf{eventi:} contiene tutti gli eventi che avranno luogo durante la fiera;
	\item \textbf{biglietti:} contiene la lista di tutti i biglietti acquistati;
	\item \textbf{padiglioni:} contiene l'elenco dei padiglioni presenti alla fiera.
\end{itemize}

\section{Accessibilità e test}
Per garantire che il sito sia accessibile al maggior numero di utenti e dispositivi è stato realizzato tenendo in considerazione le linee guida del WCAG. \newline

\subsection{Immagini}
L'uso di immagini come contenuto è stato estremamente limitato. In ogni caso per ognina delle immagini utilizzate è stato fornito un attributo \textit{alt} non vuoto contenente una descrizione dell'immagine. Il corretto funzionamento degli attributi \textit{alt} è stato verificato usando l'estensione per chrome \textit{Image Alt Text Viewer}.

\subsection{Colori e contrasto}
La scelta dei colori per il sito è stata fatta tenendo in considerazione la presenza di diffusi problemi visivi tra la popolazione, e quindi probabilmente anche tra gli utenti. I colori sono stati scelti in modo da garantire sempre un contrasto elevato tra le varie parti \textbf{TODO}

\subsection{Markup e fogli di stile}
Per garantire la massima compatibilità possibile con i vari dispositivi il sito è stato realizzato rispettando lo standard XHTML Strict e le linee guida W3C. Tutte le pagine sono state validate utilizzando il validatore del W3C (\url{https://validator.w3.org}). \newline
I fogli di stile sono stati realizzati utilizzando CSS puri e validati usando il sito \url{https://jigsaw.w3.org/css-validator/}.

\subsection{Lingua}
La lingua principale del sito è l'italiano. Nei casi in cui sono stati inseriti termini non appartenenti alla lingua italiana essi sono stati rinchiusi in un marcatore che ne dichiarasse la lingua di appartenenza, ad esempio <span xml:lang="en">...</span>. Questo accorgimento serve per permettere agli screen reader di leggere tali termini utilizzando la corretta pronuncia.

\subsection{Tabelle}
Nel sito è stata utilizzata una sola tabella nella pagina \textit{Controllo Ricavi}, destinata al solo amministratore. \textbf{TODO} non so cosa hai fatto

\subsection{Compatibilità e portabilità}
Il sito è stato progettato e realizzato con l'obiettivo di essere usabile dal maggior numero di utenti possibile, anche se non provvisti di browser di ultima generazione. 
Sono stati effettuati test con i seguenti browser desktop:
\begin{itemize}
	\item Internet Explorer 8;
	\item Internet Explorer 11;
	\item Edge versione 25;
	\item Chrome versione 52;
	\item Mozilla Firefox versione 41.
\end{itemize}
E sui seguenti browser mobile:
\begin{itemize}
	\item Silk versione 52 (browser Kindle Fire);
	\item Chrome per Android versione 52.
\end{itemize}
Inoltre, è stata testata la navigazione utilizzando il browser testuale Lynx.

\subsection{Dinamicità}
Per evitare di creare problemi o disturbo agli utenti con testi o immagini in movimento tutti i contenuti del sito sono statici.

\subsection{Indipendenza dal dispositivo}
Il sito è stato progettato e realizzato con l'obiettivo di essere agevolmente navigabile sia tramite computer che tramite dispositivi mobile (smartphone e tablet).
I membri del gruppo hanno testato il sito utilizzando tutti i dispositivi in nostro possesso ed è risultato visitabile in tutti i casi. \newline
In ogni caso la struttura di HTML e CSS è stata realizzata seguendo le linee guida di WCAG per l'accessibilità, quindi è ragionevole supporre che il sito sia visitabile con qualsiasi dispositivo.

\subsection{Contesto e aiuto per l'orientamento}
Anche se la struttura del sito è molto semplice, in ogni pagina è presente una \textit{breadcrumb} che permette all'utente di sapere sempre in che posizione si trova all'interno del sito. Inoltre, il titolo di ogni pagina identifica chiaramente la stessa. \newline
All'interno del contenuto sono stati inseriti dei link, dove necessario, per permettere all'utente di spostarsi agevolmente all'interno del sito. Ad esempio nelle pagine in cui è richiesta l'autenticazione per svolgere alcune operazioni è presente, accanto alla spiegazione della necessità dell'autenticazione, un collegamento alla pagina di login.

\subsection{Link}
\textbf{TODO}

\subsection{Screen reader}
Nell'analisi delle caratteristiche degli utenti a cui il sito è indirizzato non si è ritenuta significativa la fascia di utenti non vedenti, in quanto la saga di Star Wars deve buona parte del suo successo all'impatto visivo ed agli effetti speciali. \newline
Tuttavia, il sito è risultato perfettamente usabile tramite browser testuale, ed è stato realizzato seguendo le linee guida di accessibilità, quindi ci si aspetta che risulti navigabile anche utilizzando uno screen reader.


\end{document}
