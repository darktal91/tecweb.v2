\documentclass[12pt,a4paper]{article}

\usepackage{lmodern}
\usepackage[T1]{fontenc}
\usepackage[utf8]{inputenc}
\usepackage[italian]{babel}

\usepackage{hyperref}
\usepackage{graphicx}
\graphicspath{{img/}}
\usepackage{float}

\begin{document}
\begin{titlepage}
	\centering
	\vspace*{4.5cm}
	{\scshape\LARGE \textbf{EmpireCon} \par}
	\vspace{0.5cm}
	{\Large \textbf{\textit{Relazione sul progetto per il corso di Tecnologie Web}}\par}
	\vspace{1cm}
	{\Large Andrea Cardin, 1030310\par}
	{\Large Andrea Nalesso, matricola\par}
	{\Large Ismaele Gobbo, matricola\par}
	\vspace*{\fill}
\end{titlepage}

\tableofcontents
\newpage

% Come inserire una sezione
% \section{Nome sezione}
% \input{sections/nome file latex della sezione senza .tex}
% \newpage
% Esempio:
% \section{Introduzione}
% placeholder
% \newpage


\end{document}
